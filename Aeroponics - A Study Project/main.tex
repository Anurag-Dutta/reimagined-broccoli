\documentclass{article}
\usepackage[utf8]{inputenc}
\usepackage{graphicx}
\begin{document}
\begin{Huge}
\begin{center}
Aeroponics
\end{center}
\end{Huge}
\section{Introduction}
Aeroponics is the process of growing plants in an air or mist environment without the use of soil or an aggregate medium.\\
In other words, it is the whole plant, roots and all, are suspended in midair.\\
The word Aeroponics is derived from two Latin words aero (meaning air) and ponos[1] (meaning labour).
The basic principle of aeroponic growing is to grow plants suspended in a closed or semi-closed environment by spraying the plant's dangling roots and lower stem with an atomized or sprayed, nutrient-rich water solution. The leaves and crown, often called the canopy, extend above. The roots of the plant are separated by the plant support structure. Often, closed-cell foam is compressed around the lower stem and inserted into an opening in the aeroponic chamber, which decreases labor and expense; for larger plants, trellising is used to suspend the weight of vegetation and fruit.\\
Ideally, the environment is kept free from pests and disease so that the plants may grow healthier and more quickly than plants grown in a medium. However, since most aeroponic environments are not perfectly closed off to the outside, pests and disease may still cause a threat. Controlled environments advance plant development, health, growth, flowering and fruiting for any given plant species and cultivars.
Due to the sensitivity of root systems, aeroponics is often combined with conventional hydroponics, which is used as an emergency "crop saver" – backup nutrition and water supply – if the aeroponic apparatus fails.
High-pressure aeroponics is defined as delivering nutrients to the roots via 20–50 micrometer mist heads using a high-pressure (80 pounds per square inch (550 kPa))
\section{Types}
\subsection{Low-pressure units}
In most low-pressure aeroponic[2] gardens, the plant roots are suspended above a reservoir of nutrient solution or inside a channel connected to a reservoir. A low-pressure pump delivers nutrient solution via jets or by ultrasonic transducers, which then drips or drains back into the reservoir. As plants grow to maturity in these units, they tend to suffer from dry sections of the root systems, which prevent adequate nutrient uptake. These units, because of cost, lack features to purify the nutrient solution, and adequately remove incontinuities, debris, and unwanted pathogens. Such units are usually suitable for bench top growing and demonstrating the principles of aeroponics.
\subsection{High-pressure devices}
High-pressure aeroponic techniques, where the mist is generated by high-pressure pump(s), are typically used in the cultivation of high value crops and plant specimens that can offset the high setup costs associated with this method of horticulture.\\
High-pressure aeroponics systems include technologies for air and water purification, nutrient sterilization, low-mass polymers and pressurized nutrient delivery systems.
\subsection{Commercial systems}
Commercial aeroponic systems comprise high-pressure device hardware and biological systems. The biological systems matrix includes enhancements for extended plant life and crop maturation.\\
Biological subsystems and hardware components include effluent controls systems, disease prevention, pathogen resistance features, precision timing and nutrient solution pressurization, heating and cooling sensors, thermal control of solutions, efficient photon-flux light arrays[3], spectrum filtration spanning, fail-safe sensors and protection, reduced maintenance and labor-saving features, and ergonomics and long-term reliability features.\\
Commercial aeroponic systems, like the high-pressure devices, are used for the cultivation of high value crops where multiple crop rotations are achieved on an ongoing commercial basis.\\
Advanced commercial systems include data gathering, monitoring, analytical feedback and internet connections to various subsystems.
\section{History}
\subsection{Genesis Machine, 1983}
The first commercially available aeroponic apparatus was manufactured and marketed by GTi in 1983. GTi's device incorporated an open-loop water driven apparatus, controlled by a microchip, and delivered a high pressure, hydro-atomized nutrient spray inside an aeroponic chamber. The Genesis Machine connected to a water faucet and an electrical outlet.
\subsection{Genesis Growing System, 1985}
By 1985, GTi introduced second generation aeroponics hardware, known as the "Genesis Growing System". This second generation aeroponic apparatus was a closed-loop system. It utilized recycled effluent precisely controlled by a microprocessor. Aeroponics graduated to the capability of supporting seed germination, thus making GTi's[4] the world's first plant and harvest aeroponic system.
Many of these open-loop unit and closed-loop aeroponic systems are still in operation today.
\section{Advantages}
\subsection{Greater control of plant environment}
Aeroponics allows more control of the environment around the root zone, as, unlike other plant growth systems, the plant roots are not constantly surrounded by some medium (as, for example, with hydroponics, where the roots are constantly immersed in water).
\subsection{Improved nutrient feeding}
A variety of different nutrient solutions can be administered to the root zone using aeroponics without needing to flush out any solution or matrix in which the roots had previously been immersed. This elevated level of control would be useful when researching the effect of a varied regimen of nutrient application to the roots of a plant species of interest. In a similar manner, aeroponics allows a greater range of growth conditions than other nutrient delivery systems. The interval and duration of the nutrient spray, for example, can be very finely attuned to the needs of a specific plant species. The aerial tissue can be subjected to a completely different environment from that of the roots.
\subsection{More cost effective}
Aeroponic systems are more cost effective than other systems. Because of the reduced volume of solution throughput (discussed above), less water and fewer nutrients are needed in the system at any given time compared to other nutrient delivery systems. The need for substrates is also eliminated, as is the need for many moving parts .
\subsection{More cost effective}
Aeroponic systems are more cost effective than other systems. Because of the reduced volume of solution throughput (discussed above), less water and fewer nutrients are needed in the system at any given time compared to other nutrient delivery systems. The need for substrates is also eliminated, as is the need for many moving parts.
\subsection{More user friendly}
The design of an aeroponic system allows ease of working with the plants. This results from the separation of the plants from each other, and the fact that the plants are suspended in air and the roots are not entrapped in any kind of matrix. Consequently, the harvesting of individual plants is quite simple and straightforward. Likewise, removal of any plant that may be infected with some type of pathogen is easily accomplished without risk of uprooting or contaminating nearby plants.
\section{Case Studies}
Aeroponics eventually left the laboratories and entered into the commercial cultivation arena. In 1966, commercial aeroponic pioneer B. Briggs succeeded in inducing roots on hardwood cuttings by air-rooting. Briggs discovered that air-rooted cuttings were tougher and more hardened than those formed in soil and concluded that the basic principle of air-rooting is sound. He discovered air-rooted trees could be transplanted to soil without suffering from transplant shock or setback to normal growth. Transplant shock is normally observed in hydroponic transplants. 
In Israel in 1982, L. Nir[5] developed a patent for an aeroponic apparatus using compressed low-pressure air to deliver a nutrient solution to suspended plants, held by Styrofoam, inside large metal containers. 
In summer 1976, British researcher John Prewer carried out a series of aeroponic experiments near New Port, Isle of Wight, U.K., in which lettuces (variety Tom Thumb) were grown from seed to maturity in 22 days in polyethylene film tubes made rigid by pressurized air supplied by ventilating fans. The equipment used to convert the water-nutrient into fog droplets was supplied by Mee Industries of California. "In 1984 in association with John Prewer, a commercial grower on the Isle of Wight - Kings Nurseries - used a different design of aeroponics system to grow strawberry plants. The plants flourished and produced a heavy crop of strawberries which were picked by the nursery's customers. The system proved particularly popular with elderly customers who appreciated the cleanliness, quality and flavor of the strawberries, and the fact they did not have to stoop when picking the fruit."
In 1983, R. Stoner[6] filed a patent for the first microprocessor interface to deliver tap water and nutrients into an enclosed aeroponic chamber made of plastic. Stoner has gone on to develop numerous companies researching and advancing aeroponic hardware, interfaces, bio controls and components for commercial aeroponic crop production. 
In 1985, Stoner's company, GTi, was the first company to manufacture, market and apply large-scale closed-loop aeroponic systems into greenhouses for commercial crop production. 
In the 1990s, GHE or General Hydroponics [Europe] thought to try to introduce aeroponics to the hobby hydroponics market and finally produced the Aero Garden system. However, this could not be classed as 'true' aeroponics because the Aero Garden produced tiny droplets of solution rather than a fine mist of solution; the fine mist was meant to reproduce true Amazon rain. In any case, a product was introduced to the market and the grower could broadly claim to be growing their hydroponic produce aeroponically. A demand for aeroponic growing in the hobby market had been established and moreover it was thought of as the ultimate hydroponic growing technique. The difference between true aeroponic mist growing and aeroponic droplet growing had become very blurred in the eyes of many people. At the end of the nineties, a UK firm, Nutriculture, was encouraged enough by industry talk to trial true aeroponic growing; although these trials showed positive results compared with more traditional growing techniques such as Nutrient Film Technique (NFT) and Ebb and Flood there were drawbacks, namely cost and maintenance. To accomplish true mist aeroponics a special pump had to be used which also presented scalability problems. Droplet-aeroponics was easier to manufacture, and as it produced comparable results to mist-aeroponics, Nutriculture began development of a scalable, easy to use droplet-aeroponic system. Through trials they found that aeroponics was ideal for plant propagation; plants could be propagated without medium and could even be grown-on. In the end, Nutriculture acknowledged that better results could be achieved if the plant was propagated in their branded X-stream aeroponic propagator and moved on to a specially designed droplet-aeroponic growing system - the Amazon.
\section{Conclusion}
To sum it up, the best benefits of aeroponics are the massive plant’s growth and higher yields compared to other systems. However, these advantages also come with a cost. The price to set up the system is quite expensive and it also requires technical expertise as well as advanced knowledge (about pH and nutrient density ratio) for this special kind of plant cultivation. International organisations like NASA, WHO and many more are well adapted to this technique. They are now working on the scalable propagation of this technique in various domains. This method of growing plant will surely prove to be a great boon for human kind in near future.
\section{References}
1.	Peterson, B. J., S. E. Burnett, O. Sanchez. (2018). ‘Sub-mist is effective for propagation of Korean lilac and inkberry by stem cuttings’. HortTechnology. 28(3):378–381.\\
2.	Briggs, B.A. (1966). An experiment in air-rooting. International Plant Propagators' Society.\\
3.	Nir, I. (1982), Apparatus and Method for Plant growth in Aeroponic Conditions., Patent United States\\
4.	The system employed is described in detail in UK patent No.1 600 477 (filed 12 November 1976 - Complete Specification published 14 October 1981 - title IMPROVEMENTS IN AND RELATING TO THE PROPAGATION OF PLANTS).\\
5.	Stoner, R.J. and J.M. Clawson (1999–2000). Low-mass, Inflatable Aeroponic System for High Performance Food Production. Principal Investigator, NASA SBIR NAS10-00017\\
6.	T.W. Halstead and T.K. Scott (1990). Experiments of plants in space. In Fundamentals of space biology, M. Asashima and G.M. Malacinski (eds.), pp. 9-19. Springer-Verlag.
\end{document}
